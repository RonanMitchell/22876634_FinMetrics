\documentclass[11pt,preprint, authoryear]{elsarticle}

\usepackage{lmodern}
%%%% My spacing
\usepackage{setspace}
\setstretch{1.2}
\DeclareMathSizes{12}{14}{10}{10}

% Wrap around which gives all figures included the [H] command, or places it "here". This can be tedious to code in Rmarkdown.
\usepackage{float}
\let\origfigure\figure
\let\endorigfigure\endfigure
\renewenvironment{figure}[1][2] {
    \expandafter\origfigure\expandafter[H]
} {
    \endorigfigure
}

\let\origtable\table
\let\endorigtable\endtable
\renewenvironment{table}[1][2] {
    \expandafter\origtable\expandafter[H]
} {
    \endorigtable
}


\usepackage{ifxetex,ifluatex}
\usepackage{fixltx2e} % provides \textsubscript
\ifnum 0\ifxetex 1\fi\ifluatex 1\fi=0 % if pdftex
  \usepackage[T1]{fontenc}
  \usepackage[utf8]{inputenc}
\else % if luatex or xelatex
  \ifxetex
    \usepackage{mathspec}
    \usepackage{xltxtra,xunicode}
  \else
    \usepackage{fontspec}
  \fi
  \defaultfontfeatures{Mapping=tex-text,Scale=MatchLowercase}
  \newcommand{\euro}{€}
\fi

\usepackage{amssymb, amsmath, amsthm, amsfonts}

\def\bibsection{\section*{References}} %%% Make "References" appear before bibliography


\usepackage[round]{natbib}

\usepackage{longtable}
\usepackage[margin=2.3cm,bottom=2cm,top=2.5cm, includefoot]{geometry}
\usepackage{fancyhdr}
\usepackage[bottom, hang, flushmargin]{footmisc}
\usepackage{graphicx}
\numberwithin{equation}{section}
\numberwithin{figure}{section}
\numberwithin{table}{section}
\setlength{\parindent}{0cm}
\setlength{\parskip}{1.3ex plus 0.5ex minus 0.3ex}
\usepackage{textcomp}
\renewcommand{\headrulewidth}{0.2pt}
\renewcommand{\footrulewidth}{0.3pt}

\usepackage{array}
\newcolumntype{x}[1]{>{\centering\arraybackslash\hspace{0pt}}p{#1}}

%%%%  Remove the "preprint submitted to" part. Don't worry about this either, it just looks better without it:
\makeatletter
\def\ps@pprintTitle{%
  \let\@oddhead\@empty
  \let\@evenhead\@empty
  \let\@oddfoot\@empty
  \let\@evenfoot\@oddfoot
}
\makeatother

 \def\tightlist{} % This allows for subbullets!

\usepackage{hyperref}
\hypersetup{breaklinks=true,
            bookmarks=true,
            colorlinks=true,
            citecolor=blue,
            urlcolor=blue,
            linkcolor=blue,
            pdfborder={0 0 0}}


% The following packages allow huxtable to work:
\usepackage{siunitx}
\usepackage{multirow}
\usepackage{hhline}
\usepackage{calc}
\usepackage{tabularx}
\usepackage{booktabs}
\usepackage{caption}


\newenvironment{columns}[1][]{}{}

\newenvironment{column}[1]{\begin{minipage}{#1}\ignorespaces}{%
\end{minipage}
\ifhmode\unskip\fi
\aftergroup\useignorespacesandallpars}

\def\useignorespacesandallpars#1\ignorespaces\fi{%
#1\fi\ignorespacesandallpars}

\makeatletter
\def\ignorespacesandallpars{%
  \@ifnextchar\par
    {\expandafter\ignorespacesandallpars\@gobble}%
    {}%
}
\makeatother

\newenvironment{CSLReferences}[2]{%
}

\urlstyle{same}  % don't use monospace font for urls
\setlength{\parindent}{0pt}
\setlength{\parskip}{6pt plus 2pt minus 1pt}
\setlength{\emergencystretch}{3em}  % prevent overfull lines
\setcounter{secnumdepth}{5}

%%% Use protect on footnotes to avoid problems with footnotes in titles
\let\rmarkdownfootnote\footnote%
\def\footnote{\protect\rmarkdownfootnote}
\IfFileExists{upquote.sty}{\usepackage{upquote}}{}

%%% Include extra packages specified by user

%%% Hard setting column skips for reports - this ensures greater consistency and control over the length settings in the document.
%% page layout
%% paragraphs
\setlength{\baselineskip}{12pt plus 0pt minus 0pt}
\setlength{\parskip}{12pt plus 0pt minus 0pt}
\setlength{\parindent}{0pt plus 0pt minus 0pt}
%% floats
\setlength{\floatsep}{12pt plus 0 pt minus 0pt}
\setlength{\textfloatsep}{20pt plus 0pt minus 0pt}
\setlength{\intextsep}{14pt plus 0pt minus 0pt}
\setlength{\dbltextfloatsep}{20pt plus 0pt minus 0pt}
\setlength{\dblfloatsep}{14pt plus 0pt minus 0pt}
%% maths
\setlength{\abovedisplayskip}{12pt plus 0pt minus 0pt}
\setlength{\belowdisplayskip}{12pt plus 0pt minus 0pt}
%% lists
\setlength{\topsep}{10pt plus 0pt minus 0pt}
\setlength{\partopsep}{3pt plus 0pt minus 0pt}
\setlength{\itemsep}{5pt plus 0pt minus 0pt}
\setlength{\labelsep}{8mm plus 0mm minus 0mm}
\setlength{\parsep}{\the\parskip}
\setlength{\listparindent}{\the\parindent}
%% verbatim
\setlength{\fboxsep}{5pt plus 0pt minus 0pt}



\begin{document}



\begin{frontmatter}  %

\title{Question 4}

% Set to FALSE if wanting to remove title (for submission)




\author[Add1]{Ronan Morris}
\ead{22876634}





\address[Add1]{Stellenbosch University}



\vspace{1cm}





\vspace{0.5cm}

\end{frontmatter}

\setcounter{footnote}{0}



%________________________
% Header and Footers
%%%%%%%%%%%%%%%%%%%%%%%%%%%%%%%%%
\pagestyle{fancy}
\chead{}
\rhead{}
\lfoot{}
\rfoot{\footnotesize Page \thepage}
\lhead{}
%\rfoot{\footnotesize Page \thepage } % "e.g. Page 2"
\cfoot{}

%\setlength\headheight{30pt}
%%%%%%%%%%%%%%%%%%%%%%%%%%%%%%%%%
%________________________

\headsep 35pt % So that header does not go over title




\hypertarget{past-and-future-performance}{%
\section{Past and Future
Performance}\label{past-and-future-performance}}

My first goal was to show whether past performance is a good indicator
of future performance. I wanted to show this in an intuitive way. I
think plotting the distribution of past winners' returns against the
distribution of those same past winners' subsequent years' returns is a
good idea, because likely you will see a sort of reversion to the mean.
I only used actively managed funds that showed valid returns over the
entire time period selected. I believe that a three year average of
returns, and selecting those winners that performed in the top quantile
of these returns (top 20\% of 3 year average returns) ans then seeing
how these same funds performed in the three years after their winning
periods, was a good methodology.

For this, I required a function that could be applied to all dates in
the data set, after 2006 (the first period that three year average
returns would become available). I created WinnersComparison(), a
function that takes as its input a data set, a start date, middle date,
and end date. Between start and middle, the three year average returns
are calculated, and during middle and end, the subsequent three year
returns of the same funds is calculated. This can then be mapped onto
all dates in the provided data set by lapply(). I only piped into this
function actively managed funds, because Indices track, and as such it
is not necessarily as ``lauded'' if they over-perform, as this is not
their goal. I also created LosersComparison(), which simply creates the
same data set as the winners function, except it selects and applies
this for those performers who had 3 year average returns in the bottom
20\% at any arbitrary date.

In figure 4.1 we see that winners do indeed visibly outperform their
other years (which, due to the functions mentioned, is true across all
dates). This means that on average, it is not true that winners with
good returns distributions for three years on average tend to follow
this up with another three year average of good returns. They revert to
the mean, and we can conclude that above average three year performance
is not a good indicator of the following three years also being good.
This can also imply that if someone were to change their strategy, and
put their money with such a ``winner'', they will more than likely
experience a relative loss considering that the ``winner'' will probably
just revert to the mean.

\begin{figure}[H]

{\centering \includegraphics{Question-4_files/figure-latex/unnamed-chunk-1-1} 

}

\caption{ \label{Figure4.1}}\label{fig:unnamed-chunk-1}
\end{figure}

Figure 4.2 shows similar information, except that, across all dates and
funds, losers also tend to revert back to the mean. This means that if
someone performs in the bottom 20\% for a three year average, they
likely will simply revert to the mean and perform better in the
subsequent three years. What should be learned from this is that the
past performance is generally, in either direction, not a good indicator
of future success or failure. The investment strategy should not merely
depend on whether someone has succeeded or failed in the past. So the
question becomes now, do investors know this? Do funds flow more to
funds that have done well, and flow away from those that have done
badly?

\begin{figure}[H]

{\centering \includegraphics{Question-4_files/figure-latex/unnamed-chunk-2-1} 

}

\caption{ \label{Figure4.2}}\label{fig:unnamed-chunk-2}
\end{figure}

Figure 4.3 shows a mixed opinion on this front. There is indeed a small
correlation between past average returns (over 3 years) and receiving
flows today. I fitted a line of best fit across the data, which can be
viewed. Similarly, I created four quadrants on the plane. Essentially,
this can be split into where returns are above (below) the median, and
where flows are above (below) zero. Each quadrant sees between 20-30\%
of the data, which implies that though there might be a correlation, it
is weak - as it should be.

\begin{figure}[H]

{\centering \includegraphics{Question-4_files/figure-latex/unnamed-chunk-3-1} 

}

\caption{ \label{Figure4.3}}\label{fig:unnamed-chunk-3}
\end{figure}

\bibliography{Tex/ref}





\end{document}
