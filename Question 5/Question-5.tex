\documentclass[11pt,preprint, authoryear]{elsarticle}

\usepackage{lmodern}
%%%% My spacing
\usepackage{setspace}
\setstretch{1.2}
\DeclareMathSizes{12}{14}{10}{10}

% Wrap around which gives all figures included the [H] command, or places it "here". This can be tedious to code in Rmarkdown.
\usepackage{float}
\let\origfigure\figure
\let\endorigfigure\endfigure
\renewenvironment{figure}[1][2] {
    \expandafter\origfigure\expandafter[H]
} {
    \endorigfigure
}

\let\origtable\table
\let\endorigtable\endtable
\renewenvironment{table}[1][2] {
    \expandafter\origtable\expandafter[H]
} {
    \endorigtable
}


\usepackage{ifxetex,ifluatex}
\usepackage{fixltx2e} % provides \textsubscript
\ifnum 0\ifxetex 1\fi\ifluatex 1\fi=0 % if pdftex
  \usepackage[T1]{fontenc}
  \usepackage[utf8]{inputenc}
\else % if luatex or xelatex
  \ifxetex
    \usepackage{mathspec}
    \usepackage{xltxtra,xunicode}
  \else
    \usepackage{fontspec}
  \fi
  \defaultfontfeatures{Mapping=tex-text,Scale=MatchLowercase}
  \newcommand{\euro}{€}
\fi

\usepackage{amssymb, amsmath, amsthm, amsfonts}

\def\bibsection{\section*{References}} %%% Make "References" appear before bibliography


\usepackage[round]{natbib}

\usepackage{longtable}
\usepackage[margin=2.3cm,bottom=2cm,top=2.5cm, includefoot]{geometry}
\usepackage{fancyhdr}
\usepackage[bottom, hang, flushmargin]{footmisc}
\usepackage{graphicx}
\numberwithin{equation}{section}
\numberwithin{figure}{section}
\numberwithin{table}{section}
\setlength{\parindent}{0cm}
\setlength{\parskip}{1.3ex plus 0.5ex minus 0.3ex}
\usepackage{textcomp}
\renewcommand{\headrulewidth}{0.2pt}
\renewcommand{\footrulewidth}{0.3pt}

\usepackage{array}
\newcolumntype{x}[1]{>{\centering\arraybackslash\hspace{0pt}}p{#1}}

%%%%  Remove the "preprint submitted to" part. Don't worry about this either, it just looks better without it:
\makeatletter
\def\ps@pprintTitle{%
  \let\@oddhead\@empty
  \let\@evenhead\@empty
  \let\@oddfoot\@empty
  \let\@evenfoot\@oddfoot
}
\makeatother

 \def\tightlist{} % This allows for subbullets!

\usepackage{hyperref}
\hypersetup{breaklinks=true,
            bookmarks=true,
            colorlinks=true,
            citecolor=blue,
            urlcolor=blue,
            linkcolor=blue,
            pdfborder={0 0 0}}


% The following packages allow huxtable to work:
\usepackage{siunitx}
\usepackage{multirow}
\usepackage{hhline}
\usepackage{calc}
\usepackage{tabularx}
\usepackage{booktabs}
\usepackage{caption}


\newenvironment{columns}[1][]{}{}

\newenvironment{column}[1]{\begin{minipage}{#1}\ignorespaces}{%
\end{minipage}
\ifhmode\unskip\fi
\aftergroup\useignorespacesandallpars}

\def\useignorespacesandallpars#1\ignorespaces\fi{%
#1\fi\ignorespacesandallpars}

\makeatletter
\def\ignorespacesandallpars{%
  \@ifnextchar\par
    {\expandafter\ignorespacesandallpars\@gobble}%
    {}%
}
\makeatother

\newenvironment{CSLReferences}[2]{%
}

\urlstyle{same}  % don't use monospace font for urls
\setlength{\parindent}{0pt}
\setlength{\parskip}{6pt plus 2pt minus 1pt}
\setlength{\emergencystretch}{3em}  % prevent overfull lines
\setcounter{secnumdepth}{5}

%%% Use protect on footnotes to avoid problems with footnotes in titles
\let\rmarkdownfootnote\footnote%
\def\footnote{\protect\rmarkdownfootnote}
\IfFileExists{upquote.sty}{\usepackage{upquote}}{}

%%% Include extra packages specified by user

%%% Hard setting column skips for reports - this ensures greater consistency and control over the length settings in the document.
%% page layout
%% paragraphs
\setlength{\baselineskip}{12pt plus 0pt minus 0pt}
\setlength{\parskip}{12pt plus 0pt minus 0pt}
\setlength{\parindent}{0pt plus 0pt minus 0pt}
%% floats
\setlength{\floatsep}{12pt plus 0 pt minus 0pt}
\setlength{\textfloatsep}{20pt plus 0pt minus 0pt}
\setlength{\intextsep}{14pt plus 0pt minus 0pt}
\setlength{\dbltextfloatsep}{20pt plus 0pt minus 0pt}
\setlength{\dblfloatsep}{14pt plus 0pt minus 0pt}
%% maths
\setlength{\abovedisplayskip}{12pt plus 0pt minus 0pt}
\setlength{\belowdisplayskip}{12pt plus 0pt minus 0pt}
%% lists
\setlength{\topsep}{10pt plus 0pt minus 0pt}
\setlength{\partopsep}{3pt plus 0pt minus 0pt}
\setlength{\itemsep}{5pt plus 0pt minus 0pt}
\setlength{\labelsep}{8mm plus 0mm minus 0mm}
\setlength{\parsep}{\the\parskip}
\setlength{\listparindent}{\the\parindent}
%% verbatim
\setlength{\fboxsep}{5pt plus 0pt minus 0pt}



\begin{document}



\begin{frontmatter}  %

\title{Question 5}

% Set to FALSE if wanting to remove title (for submission)




\author[Add1]{Ronan Morris}
\ead{22876634}





\address[Add1]{Stellenbosch University}



\vspace{1cm}





\vspace{0.5cm}

\end{frontmatter}

\setcounter{footnote}{0}



%________________________
% Header and Footers
%%%%%%%%%%%%%%%%%%%%%%%%%%%%%%%%%
\pagestyle{fancy}
\chead{}
\rhead{}
\lfoot{}
\rfoot{\footnotesize Page \thepage}
\lhead{}
%\rfoot{\footnotesize Page \thepage } % "e.g. Page 2"
\cfoot{}

%\setlength\headheight{30pt}
%%%%%%%%%%%%%%%%%%%%%%%%%%%%%%%%%
%________________________

\headsep 35pt % So that header does not go over title




\hypertarget{volatility-and-garch}{%
\section{Volatility and GARCH}\label{volatility-and-garch}}

The way that I interpreted the question is three fold:

\begin{enumerate}
\def\labelenumi{\arabic{enumi}.}
\item
  I need to rank, by some metric, the average volatility of countries
  over my selected time period of the last ten years (after 2010/01/01).
  From this, I need to see where South Africa ranks.
\item
  Then, I should build a GARCH model (I have chosen univariate), select
  the best specification, and show the ``cleaned'' volatility of the
  ZAR, perhaps to be used in later explanations.
\item
  I should isolate low volatility time periods in the US dollar, and
  plot the ZAR volatility against some sort of global average
  volatility, perhaps only of selected countries, or all, and indicate
  on the same set of axes the low USD volatility periods.
\end{enumerate}

In terms of the first point, see the table ranking the top ten countries
by average volatility over the last ten years, below.

\begin{longtable}[]{@{}lr@{}}
\toprule\noalign{}
Country & Vol \\
\midrule\noalign{}
\endhead
\bottomrule\noalign{}
\endlastfoot
Ghana & 0.0001859 \\
Russia & 0.0001070 \\
Brazil & 0.0000988 \\
SouthAfrica & 0.0000969 \\
Argentina & 0.0000957 \\
Nigeria & 0.0000933 \\
Zambia & 0.0000912 \\
Egypt & 0.0000834 \\
Turkey & 0.0000826 \\
Mexico & 0.0000624 \\
\end{longtable}

As you can see, South Africa ranks in at number 4. This is an incredibly
high position. The average volatility is substantially lower than Ghana
at number one, but this is an outlier as it is higher than all others in
the table by quite a distance. My comment on this is that the ZAR is
indeed one of the most volatile currencies over the decade spanning 2010
- 2021.

I would now like to build a univariate GARCH model and estimate a noise
reduced volatility estimate for the ZAR. For this I created three
functions, which are all variations of the practical offered by Nico.
Selection\_foo() selects the best model, GARCH\_Maker\_foo creates a
GARCH model, and then I created VolPlotter() to plot the cleaned
volatility, in case I wanted to repeat this process for other countries.
I did not end up doing this, though.

My chosen specification is a gjr-GARCH, because it had the lowest AIC.
The AIC of this model is -6.483637. Figure 5.1 is the cleaned volatility
of the ZAR plotted on the same set of axes as the raw volatility. You
can clearly see that the GARCH estimate is noise reduced.

\begin{figure}[H]

{\centering \includegraphics{Question-5_files/figure-latex/unnamed-chunk-3-1} 

}

\caption{ \label{Figure5.1}}\label{fig:unnamed-chunk-3}
\end{figure}

Finally, for step three, I calculated the low volatility periods for the
USD and selected the top 3 years. Selecting only the years allows for a
broader discussion and a simpler plot for the reader to interpret. I
then plotted the ZAR volatility against the global average volatility
over the time period. Selecting the global average is relatively simple,
and perhaps not always the most informative, but given the question I
interpreted that this would be the best decision.

\begin{figure}[H]

{\centering \includegraphics{Question-5_files/figure-latex/unnamed-chunk-4-1} 

}

\caption{ \label{Figure5.2}}\label{fig:unnamed-chunk-4}
\end{figure}

In figure 5.2 you can see that 2014, 2019, and 2021 were selected as the
top 3 least volatile years for the USD in the past decade. My comment on
this consists of two parts. Firstly, it is interesting that the ZAR has
become so correlated over time to the global average volatility. You
could argue that in the first half of the decade, this correlation did
not necessarily exist. Though (I have no source for this but I think we
learned it in Time Series Econometrics) I believe that co-movement is
understood to increase during periods of economic struggle, which would
explain the similar volatility during COVID-19.

Secondly, we see that all currencies seem to become less volatile when
the USD is less volatile. In 2014, we see a rapid decline in the global
average volatility, as well as the Rand. We see a similar decrease in
volatility in 2019, and after the pandemic in 2021. I would say that
there does seem to be a good relationship between the ZAR and the USD in
terms of volatility, but this is also true for the global average
volatility, so I might not necessarily argue that the Rand specifically
has a special relationship.

\bibliography{Tex/ref}





\end{document}
