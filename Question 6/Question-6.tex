\documentclass[11pt,preprint, authoryear]{elsarticle}

\usepackage{lmodern}
%%%% My spacing
\usepackage{setspace}
\setstretch{1.2}
\DeclareMathSizes{12}{14}{10}{10}

% Wrap around which gives all figures included the [H] command, or places it "here". This can be tedious to code in Rmarkdown.
\usepackage{float}
\let\origfigure\figure
\let\endorigfigure\endfigure
\renewenvironment{figure}[1][2] {
    \expandafter\origfigure\expandafter[H]
} {
    \endorigfigure
}

\let\origtable\table
\let\endorigtable\endtable
\renewenvironment{table}[1][2] {
    \expandafter\origtable\expandafter[H]
} {
    \endorigtable
}


\usepackage{ifxetex,ifluatex}
\usepackage{fixltx2e} % provides \textsubscript
\ifnum 0\ifxetex 1\fi\ifluatex 1\fi=0 % if pdftex
  \usepackage[T1]{fontenc}
  \usepackage[utf8]{inputenc}
\else % if luatex or xelatex
  \ifxetex
    \usepackage{mathspec}
    \usepackage{xltxtra,xunicode}
  \else
    \usepackage{fontspec}
  \fi
  \defaultfontfeatures{Mapping=tex-text,Scale=MatchLowercase}
  \newcommand{\euro}{€}
\fi

\usepackage{amssymb, amsmath, amsthm, amsfonts}

\def\bibsection{\section*{References}} %%% Make "References" appear before bibliography


\usepackage[round]{natbib}

\usepackage{longtable}
\usepackage[margin=2.3cm,bottom=2cm,top=2.5cm, includefoot]{geometry}
\usepackage{fancyhdr}
\usepackage[bottom, hang, flushmargin]{footmisc}
\usepackage{graphicx}
\numberwithin{equation}{section}
\numberwithin{figure}{section}
\numberwithin{table}{section}
\setlength{\parindent}{0cm}
\setlength{\parskip}{1.3ex plus 0.5ex minus 0.3ex}
\usepackage{textcomp}
\renewcommand{\headrulewidth}{0.2pt}
\renewcommand{\footrulewidth}{0.3pt}

\usepackage{array}
\newcolumntype{x}[1]{>{\centering\arraybackslash\hspace{0pt}}p{#1}}

%%%%  Remove the "preprint submitted to" part. Don't worry about this either, it just looks better without it:
\makeatletter
\def\ps@pprintTitle{%
  \let\@oddhead\@empty
  \let\@evenhead\@empty
  \let\@oddfoot\@empty
  \let\@evenfoot\@oddfoot
}
\makeatother

 \def\tightlist{} % This allows for subbullets!

\usepackage{hyperref}
\hypersetup{breaklinks=true,
            bookmarks=true,
            colorlinks=true,
            citecolor=blue,
            urlcolor=blue,
            linkcolor=blue,
            pdfborder={0 0 0}}


% The following packages allow huxtable to work:
\usepackage{siunitx}
\usepackage{multirow}
\usepackage{hhline}
\usepackage{calc}
\usepackage{tabularx}
\usepackage{booktabs}
\usepackage{caption}


\newenvironment{columns}[1][]{}{}

\newenvironment{column}[1]{\begin{minipage}{#1}\ignorespaces}{%
\end{minipage}
\ifhmode\unskip\fi
\aftergroup\useignorespacesandallpars}

\def\useignorespacesandallpars#1\ignorespaces\fi{%
#1\fi\ignorespacesandallpars}

\makeatletter
\def\ignorespacesandallpars{%
  \@ifnextchar\par
    {\expandafter\ignorespacesandallpars\@gobble}%
    {}%
}
\makeatother

\newenvironment{CSLReferences}[2]{%
}

\urlstyle{same}  % don't use monospace font for urls
\setlength{\parindent}{0pt}
\setlength{\parskip}{6pt plus 2pt minus 1pt}
\setlength{\emergencystretch}{3em}  % prevent overfull lines
\setcounter{secnumdepth}{5}

%%% Use protect on footnotes to avoid problems with footnotes in titles
\let\rmarkdownfootnote\footnote%
\def\footnote{\protect\rmarkdownfootnote}
\IfFileExists{upquote.sty}{\usepackage{upquote}}{}

%%% Include extra packages specified by user

%%% Hard setting column skips for reports - this ensures greater consistency and control over the length settings in the document.
%% page layout
%% paragraphs
\setlength{\baselineskip}{12pt plus 0pt minus 0pt}
\setlength{\parskip}{12pt plus 0pt minus 0pt}
\setlength{\parindent}{0pt plus 0pt minus 0pt}
%% floats
\setlength{\floatsep}{12pt plus 0 pt minus 0pt}
\setlength{\textfloatsep}{20pt plus 0pt minus 0pt}
\setlength{\intextsep}{14pt plus 0pt minus 0pt}
\setlength{\dbltextfloatsep}{20pt plus 0pt minus 0pt}
\setlength{\dblfloatsep}{14pt plus 0pt minus 0pt}
%% maths
\setlength{\abovedisplayskip}{12pt plus 0pt minus 0pt}
\setlength{\belowdisplayskip}{12pt plus 0pt minus 0pt}
%% lists
\setlength{\topsep}{10pt plus 0pt minus 0pt}
\setlength{\partopsep}{3pt plus 0pt minus 0pt}
\setlength{\itemsep}{5pt plus 0pt minus 0pt}
\setlength{\labelsep}{8mm plus 0mm minus 0mm}
\setlength{\parsep}{\the\parskip}
\setlength{\listparindent}{\the\parindent}
%% verbatim
\setlength{\fboxsep}{5pt plus 0pt minus 0pt}



\begin{document}



\begin{frontmatter}  %

\title{Question 6}

% Set to FALSE if wanting to remove title (for submission)




\author[Add1]{Ronan Morris}
\ead{22876634}





\address[Add1]{Stellenbosch University}



\vspace{1cm}





\vspace{0.5cm}

\end{frontmatter}

\setcounter{footnote}{0}



%________________________
% Header and Footers
%%%%%%%%%%%%%%%%%%%%%%%%%%%%%%%%%
\pagestyle{fancy}
\chead{}
\rhead{}
\lfoot{}
\rfoot{\footnotesize Page \thepage}
\lhead{}
%\rfoot{\footnotesize Page \thepage } % "e.g. Page 2"
\cfoot{}

%\setlength\headheight{30pt}
%%%%%%%%%%%%%%%%%%%%%%%%%%%%%%%%%
%________________________

\headsep 35pt % So that header does not go over title




In this question I was tasked to create a Global Balanced Index Fund
portfolio using a mix of traded global indexes. Certain constraints were
set to me by my superiors. I would like to discuss how I handled each.

The first constraint was that it should be a ``long only'' strategy. In
accordance with this, I set the lower bound to be equal to 1\%. I set
the upper bound to be 40\%, which could seem excessive, I believe that
``too much of a good thing'' is a good mantra, and we should limit
exposure to specific assets. An upper bound of 40\% seems slightly high
but it is not unheard of.

I decided to use set the weighting for equities and bonds at 60\% and
25\% respectively. I have no particular research into the optimal
weights of these instruments, and I do not believe that I could make a
better allocation other than the limit. I will say that equities tend to
have a higher volatility, but they yield a higher return (surprisingly
so, actually - Mehra \& Prescott, 1985). This allocation should be
carefully considered in the real world, but I have simply set the
constraints to the limit.

I have not used any assets that do not have returns data for at least
the last three years, and I have used quarterly re-balancing; this is in
line with the policies of my superiors, and presumably the institution I
would be working for in this scenario. I only considered data from 2010
onwards (after the financial crisis) for this analysis. The final
consideration was what length to set the lookback period. I set it to 12
months, as I saw in much of the discussions online and in the
literature, that this is a good starting point for portfolio
optimisation.

Please see the first date of the optimised portfolio for all assets,
their weights, and the method of optimisation. Namely, these include
mean variance, minimum volatility, and the Sharpe portfolio. These have
differing profiles. For example, one would want to drive for maximum
diversification if assets have similar Sharpe ratios. The different
types also make different assumptions about returns, or expected
returns.

\begin{verbatim}
## # A tibble: 15 x 5
##    stocks             mv minvol sharpe date      
##    <chr>           <dbl>  <dbl>  <dbl> <date>    
##  1 ADXY Index     0.0100 0.2    0.0769 2010-03-31
##  2 BCOMTR Index   0.01   0.0355 0.0769 2010-03-31
##  3 DXY Index      0.0100 0.2    0.0769 2010-03-31
##  4 LEATTREU Index 0.0100 0.0100 0.0769 2010-03-31
##  5 LGAGTRUH Index 0.0100 0.2    0.0769 2010-03-31
##  6 LGCPTRUH Index 0.0100 0.0100 0.0769 2010-03-31
##  7 LP05TREH Index 0.0100 0.0724 0.0769 2010-03-31
##  8 LUACTRUU Index 0.120  0.0100 0.0769 2010-03-31
##  9 LUAGTRUU Index 0.0100 0.2    0.0769 2010-03-31
## 10 MSCI_ACWI      0.200  0.01   0.0769 2010-03-31
## 11 MSCI_Jap       0.200  0.0320 0.0769 2010-03-31
## 12 MSCI_RE        0.200  0.01   0.0769 2010-03-31
## 13 MSCI_USA       0.200  0.0100 0.0769 2010-03-31
## 14 ADXY Index     0.0100 0.2    0.0769 2010-06-30
## 15 BCOMTR Index   0.01   0.0355 0.0769 2010-06-30
\end{verbatim}

\bibliography{Tex/ref}





\end{document}
